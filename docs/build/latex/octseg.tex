%% Generated by Sphinx.
\def\sphinxdocclass{report}
\documentclass[letterpaper,10pt,english]{sphinxmanual}
\ifdefined\pdfpxdimen
   \let\sphinxpxdimen\pdfpxdimen\else\newdimen\sphinxpxdimen
\fi \sphinxpxdimen=.75bp\relax

\PassOptionsToPackage{warn}{textcomp}
\usepackage[utf8]{inputenc}
\ifdefined\DeclareUnicodeCharacter
% support both utf8 and utf8x syntaxes
  \ifdefined\DeclareUnicodeCharacterAsOptional
    \def\sphinxDUC#1{\DeclareUnicodeCharacter{"#1}}
  \else
    \let\sphinxDUC\DeclareUnicodeCharacter
  \fi
  \sphinxDUC{00A0}{\nobreakspace}
  \sphinxDUC{2500}{\sphinxunichar{2500}}
  \sphinxDUC{2502}{\sphinxunichar{2502}}
  \sphinxDUC{2514}{\sphinxunichar{2514}}
  \sphinxDUC{251C}{\sphinxunichar{251C}}
  \sphinxDUC{2572}{\textbackslash}
\fi
\usepackage{cmap}
\usepackage[T1]{fontenc}
\usepackage{amsmath,amssymb,amstext}
\usepackage{babel}



\usepackage{times}
\expandafter\ifx\csname T@LGR\endcsname\relax
\else
% LGR was declared as font encoding
  \substitutefont{LGR}{\rmdefault}{cmr}
  \substitutefont{LGR}{\sfdefault}{cmss}
  \substitutefont{LGR}{\ttdefault}{cmtt}
\fi
\expandafter\ifx\csname T@X2\endcsname\relax
  \expandafter\ifx\csname T@T2A\endcsname\relax
  \else
  % T2A was declared as font encoding
    \substitutefont{T2A}{\rmdefault}{cmr}
    \substitutefont{T2A}{\sfdefault}{cmss}
    \substitutefont{T2A}{\ttdefault}{cmtt}
  \fi
\else
% X2 was declared as font encoding
  \substitutefont{X2}{\rmdefault}{cmr}
  \substitutefont{X2}{\sfdefault}{cmss}
  \substitutefont{X2}{\ttdefault}{cmtt}
\fi


\usepackage[Bjarne]{fncychap}
\usepackage{sphinx}

\fvset{fontsize=\small}
\usepackage{geometry}

% Include hyperref last.
\usepackage{hyperref}
% Fix anchor placement for figures with captions.
\usepackage{hypcap}% it must be loaded after hyperref.
% Set up styles of URL: it should be placed after hyperref.
\urlstyle{same}
\addto\captionsenglish{\renewcommand{\contentsname}{Contents:}}

\usepackage{sphinxmessages}
\setcounter{tocdepth}{1}



\title{OCTseg}
\date{Aug 02, 2019}
\release{}
\author{Mohammad Haft-Javaherian}
\newcommand{\sphinxlogo}{\vbox{}}
\renewcommand{\releasename}{}
\makeindex
\begin{document}

\pagestyle{empty}
\sphinxmaketitle
\pagestyle{plain}
\sphinxtableofcontents
\pagestyle{normal}
\phantomsection\label{\detokenize{index::doc}}



\chapter{Indices and tables}
\label{\detokenize{index:indices-and-tables}}\begin{itemize}
\item {} 
\DUrole{xref,std,std-ref}{genindex}

\item {} 
\DUrole{xref,std,std-ref}{modindex}

\item {} 
\DUrole{xref,std,std-ref}{search}

\end{itemize}


\chapter{util}
\label{\detokenize{index:util}}

\section{load data}
\label{\detokenize{index:module-util.load_data}}\label{\detokenize{index:load-data}}\index{util.load\_data (module)@\spxentry{util.load\_data}\spxextra{module}}
Convert an 2D or 3D image from polar or cylindrical coordinate to the
cartesian coordinate.
\index{im\_fix\_width() (in module util.load\_data)@\spxentry{im\_fix\_width()}\spxextra{in module util.load\_data}}

\begin{fulllineitems}
\phantomsection\label{\detokenize{index:util.load_data.im_fix_width}}\pysiglinewithargsret{\sphinxcode{\sphinxupquote{util.load\_data.}}\sphinxbfcode{\sphinxupquote{im\_fix\_width}}}{\emph{im}, \emph{w}}{}
pad or crop the 3D image to have width and length equal to the input width

\end{fulllineitems}



\section{polar to cartesian}
\label{\detokenize{index:module-util.polar2cartesian}}\label{\detokenize{index:polar-to-cartesian}}\index{util.polar2cartesian (module)@\spxentry{util.polar2cartesian}\spxextra{module}}
Convert an 2D or 3D image from polar or cylindrical coordinate to the
cartesian coordinate.


\section{process oct folder}
\label{\detokenize{index:module-util.process_oct_folder}}\label{\detokenize{index:process-oct-folder}}\index{util.process\_oct\_folder (module)@\spxentry{util.process\_oct\_folder}\spxextra{module}}
process OCT folder to generate the segmentation labels of cases. Each case all three -.PSTIF, -.INI, and -ROI.txt
files


\section{read oct roi file}
\label{\detokenize{index:module-util.read_oct_roi_file}}\label{\detokenize{index:read-oct-roi-file}}\index{util.read\_oct\_roi\_file (module)@\spxentry{util.read\_oct\_roi\_file}\spxextra{module}}
Read ROI file generated based on the and generate segmentation results.
\index{lumen\_iel\_mask() (in module util.read\_oct\_roi\_file)@\spxentry{lumen\_iel\_mask()}\spxextra{in module util.read\_oct\_roi\_file}}

\begin{fulllineitems}
\phantomsection\label{\detokenize{index:util.read_oct_roi_file.lumen_iel_mask}}\pysiglinewithargsret{\sphinxcode{\sphinxupquote{util.read\_oct\_roi\_file.}}\sphinxbfcode{\sphinxupquote{lumen\_iel\_mask}}}{\emph{obj\_list}, \emph{im\_shape}}{}
generate lumen or IEL mask based on the point list.

\end{fulllineitems}

\index{roi\_file\_parser() (in module util.read\_oct\_roi\_file)@\spxentry{roi\_file\_parser()}\spxextra{in module util.read\_oct\_roi\_file}}

\begin{fulllineitems}
\phantomsection\label{\detokenize{index:util.read_oct_roi_file.roi_file_parser}}\pysiglinewithargsret{\sphinxcode{\sphinxupquote{util.read\_oct\_roi\_file.}}\sphinxbfcode{\sphinxupquote{roi\_file\_parser}}}{\emph{file\_path}}{}
Parse roi file and output the lists of objects

\end{fulllineitems}



\chapter{unet}
\label{\detokenize{index:unet}}

\section{loss}
\label{\detokenize{index:module-unet.loss}}\label{\detokenize{index:loss}}\index{unet.loss (module)@\spxentry{unet.loss}\spxextra{module}}
CNN related loss functions
\index{dice\_loss() (in module unet.loss)@\spxentry{dice\_loss()}\spxextra{in module unet.loss}}

\begin{fulllineitems}
\phantomsection\label{\detokenize{index:unet.loss.dice_loss}}\pysiglinewithargsret{\sphinxcode{\sphinxupquote{unet.loss.}}\sphinxbfcode{\sphinxupquote{dice\_loss}}}{\emph{label}, \emph{target}}{}
soft Dice coefficient loss
TP, FP, and FN are true positive, false positive, and false negative.
\begin{equation*}
\begin{split}dice  &=  \frac{2 \times TP}{ 2 \times TP + FN + FP} \\
dice  &=  \frac{2 \times TP}{(TP + FN) + (TP + FP)}\end{split}
\end{equation*}
objective is to maximize the dice, thus the loss is negate of dice for numerical stability (+1 in denominator)
and fixing the loss range (+1 in numerator and +1 to the negated dice)
The final Dice loss is formulated as
\begin{equation*}
\begin{split}dice \ loss = 1 - \frac{2 \times TP + 1}{(TP + FN) + (TP + FP ) + 1}\end{split}
\end{equation*}
it is soft as each components of the confusion matrix (TP, FP, and FN) are estimated by dot product of
probability instead of hard classification
\begin{quote}\begin{description}
\item[{Parameters}] \leavevmode\begin{itemize}
\item {} 
\sphinxstyleliteralstrong{\sphinxupquote{label}} \textendash{} 4D or 5D label tensor

\item {} 
\sphinxstyleliteralstrong{\sphinxupquote{target}} \textendash{} 4D or 5d target tensor

\end{itemize}

\item[{Returns}] \leavevmode
dice loss

\end{description}\end{quote}

\end{fulllineitems}

\index{multi\_loss\_fun() (in module unet.loss)@\spxentry{multi\_loss\_fun()}\spxextra{in module unet.loss}}

\begin{fulllineitems}
\phantomsection\label{\detokenize{index:unet.loss.multi_loss_fun}}\pysiglinewithargsret{\sphinxcode{\sphinxupquote{unet.loss.}}\sphinxbfcode{\sphinxupquote{multi\_loss\_fun}}}{\emph{loss\_weight}}{}
semantic loss function based on the weighted cross entropy and dice and wighted by the loss weights in the input
argument
\begin{quote}\begin{description}
\item[{Parameters}] \leavevmode
\sphinxstyleliteralstrong{\sphinxupquote{loss\_weight}} \textendash{} a list with two weights for weighted cross entropy and dice losses, respectively.

\item[{Returns}] \leavevmode
\begin{description}
\item[{return a function, which similar to {\hyperref[\detokenize{index:unet.loss.weighted_cross_entropy}]{\sphinxcrossref{\sphinxcode{\sphinxupquote{weighted\_cross\_entropy()}}}}} and {\hyperref[\detokenize{index:unet.loss.dice_loss}]{\sphinxcrossref{\sphinxcode{\sphinxupquote{dice\_loss()}}}}}}] \leavevmode
has label and target arguments

\end{description}


\item[{Seemore}] \leavevmode
{\hyperref[\detokenize{index:unet.loss.weighted_cross_entropy}]{\sphinxcrossref{\sphinxcode{\sphinxupquote{weighted\_cross\_entropy()}}}}}, {\hyperref[\detokenize{index:unet.loss.dice_loss}]{\sphinxcrossref{\sphinxcode{\sphinxupquote{dice\_loss()}}}}} :

\end{description}\end{quote}

\end{fulllineitems}

\index{weighted\_cross\_entropy() (in module unet.loss)@\spxentry{weighted\_cross\_entropy()}\spxextra{in module unet.loss}}

\begin{fulllineitems}
\phantomsection\label{\detokenize{index:unet.loss.weighted_cross_entropy}}\pysiglinewithargsret{\sphinxcode{\sphinxupquote{unet.loss.}}\sphinxbfcode{\sphinxupquote{weighted\_cross\_entropy}}}{\emph{label}, \emph{target}}{}
weighted cross entropy with foreground pixels having ten times higher weights
\begin{quote}\begin{description}
\item[{Parameters}] \leavevmode\begin{itemize}
\item {} 
\sphinxstyleliteralstrong{\sphinxupquote{label}} \textendash{} 4D or 5D label tensor

\item {} 
\sphinxstyleliteralstrong{\sphinxupquote{target}} \textendash{} 4D or 5d target tensor

\end{itemize}

\item[{Returns}] \leavevmode
weighted cross entropy value

\item[{TODO}] \leavevmode
add positive weight as an argument

\end{description}\end{quote}

\end{fulllineitems}



\section{ops}
\label{\detokenize{index:module-unet.ops}}\label{\detokenize{index:ops}}\index{unet.ops (module)@\spxentry{unet.ops}\spxextra{module}}
CNN related operations
\index{accuracy() (in module unet.ops)@\spxentry{accuracy()}\spxextra{in module unet.ops}}

\begin{fulllineitems}
\phantomsection\label{\detokenize{index:unet.ops.accuracy}}\pysiglinewithargsret{\sphinxcode{\sphinxupquote{unet.ops.}}\sphinxbfcode{\sphinxupquote{accuracy}}}{\emph{labels}, \emph{logits}}{}
measure accuracy metrics

\end{fulllineitems}

\index{img\_aug\_carts() (in module unet.ops)@\spxentry{img\_aug\_carts()}\spxextra{in module unet.ops}}

\begin{fulllineitems}
\phantomsection\label{\detokenize{index:unet.ops.img_aug_carts}}\pysiglinewithargsret{\sphinxcode{\sphinxupquote{unet.ops.}}\sphinxbfcode{\sphinxupquote{img\_aug\_carts}}}{\emph{im}, \emph{l}}{}
Data augmentation in Cartesian

\end{fulllineitems}

\index{img\_aug\_polar() (in module unet.ops)@\spxentry{img\_aug\_polar()}\spxextra{in module unet.ops}}

\begin{fulllineitems}
\phantomsection\label{\detokenize{index:unet.ops.img_aug_polar}}\pysiglinewithargsret{\sphinxcode{\sphinxupquote{unet.ops.}}\sphinxbfcode{\sphinxupquote{img\_aug\_polar}}}{\emph{im}, \emph{l}}{}
Data augmentation in Polar coordinate

\end{fulllineitems}

\index{img\_rand\_scale() (in module unet.ops)@\spxentry{img\_rand\_scale()}\spxextra{in module unet.ops}}

\begin{fulllineitems}
\phantomsection\label{\detokenize{index:unet.ops.img_rand_scale}}\pysiglinewithargsret{\sphinxcode{\sphinxupquote{unet.ops.}}\sphinxbfcode{\sphinxupquote{img\_rand\_scale}}}{\emph{im}, \emph{scale}, \emph{order}}{}
scale one image batch

\end{fulllineitems}

\index{placeholder\_inputs() (in module unet.ops)@\spxentry{placeholder\_inputs()}\spxextra{in module unet.ops}}

\begin{fulllineitems}
\phantomsection\label{\detokenize{index:unet.ops.placeholder_inputs}}\pysiglinewithargsret{\sphinxcode{\sphinxupquote{unet.ops.}}\sphinxbfcode{\sphinxupquote{placeholder\_inputs}}}{\emph{im\_shape}, \emph{outCh}}{}
Generate placeholder variables to represent the input tensors.

\end{fulllineitems}



\section{unet}
\label{\detokenize{index:id1}}\phantomsection\label{\detokenize{index:module-unet.unet}}\index{unet.unet (module)@\spxentry{unet.unet}\spxextra{module}}
Build unet model
\index{unet\_model() (in module unet.unet)@\spxentry{unet\_model()}\spxextra{in module unet.unet}}

\begin{fulllineitems}
\phantomsection\label{\detokenize{index:unet.unet.unet_model}}\pysiglinewithargsret{\sphinxcode{\sphinxupquote{unet.unet.}}\sphinxbfcode{\sphinxupquote{unet\_model}}}{\emph{im\_shape}, \emph{nFeature=32}, \emph{outCh=2}}{}
Build U-Net model.
\begin{quote}\begin{description}
\item[{Parameters}] \leavevmode\begin{itemize}
\item {} 
\sphinxstyleliteralstrong{\sphinxupquote{x}} \textendash{} input placeholder

\item {} 
\sphinxstyleliteralstrong{\sphinxupquote{outCh}} \textendash{} number of output channels

\end{itemize}

\item[{Returns}] \leavevmode
keras model

\end{description}\end{quote}

\end{fulllineitems}



\renewcommand{\indexname}{Python Module Index}
\begin{sphinxtheindex}
\let\bigletter\sphinxstyleindexlettergroup
\bigletter{u}
\item\relax\sphinxstyleindexentry{unet.loss}\sphinxstyleindexpageref{index:\detokenize{module-unet.loss}}
\item\relax\sphinxstyleindexentry{unet.ops}\sphinxstyleindexpageref{index:\detokenize{module-unet.ops}}
\item\relax\sphinxstyleindexentry{unet.unet}\sphinxstyleindexpageref{index:\detokenize{module-unet.unet}}
\item\relax\sphinxstyleindexentry{util.load\_data}\sphinxstyleindexpageref{index:\detokenize{module-util.load_data}}
\item\relax\sphinxstyleindexentry{util.polar2cartesian}\sphinxstyleindexpageref{index:\detokenize{module-util.polar2cartesian}}
\item\relax\sphinxstyleindexentry{util.process\_oct\_folder}\sphinxstyleindexpageref{index:\detokenize{module-util.process_oct_folder}}
\item\relax\sphinxstyleindexentry{util.read\_oct\_roi\_file}\sphinxstyleindexpageref{index:\detokenize{module-util.read_oct_roi_file}}
\end{sphinxtheindex}

\renewcommand{\indexname}{Index}
\printindex
\end{document}